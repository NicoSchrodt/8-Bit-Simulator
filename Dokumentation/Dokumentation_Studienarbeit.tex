



%Autor: Nico Schrodt
%Januar 2021 - März 2021


\documentclass[12pt]{article}

\usepackage{geometry}
\usepackage{blindtext}
\usepackage{setspace}
\usepackage{hyperref}
\usepackage[headsepline=0.8pt, footsepline =0.8pt]{scrlayer-scrpage}
\usepackage{listings}
\usepackage{subcaption}
\usepackage{tabularx}
\usepackage{color, colortbl}
%\usepackage{pdfpages}
\usepackage{amssymb}

\geometry{a4paper, top=25mm, left=35mm, right=25mm, bottom=25mm, headsep=13mm, footskip=12mm, head=14.5pt}

%encoding
%--------------------------------------
\usepackage[utf8]{inputenc}
\usepackage[T1]{fontenc}
%--------------------------------------

%German-specific commands
%--------------------------------------
\usepackage[ngerman]{babel}
%--------------------------------------

%Hyphenation rules
%--------------------------------------
\usepackage{hyphenat}
%--------------------------------------

\usepackage{graphicx}

\newcommand{\Autor}{Andreas Schmider, Nico Schrodt}

\newcommand{\Bearbeitungszeitraum}{2 Semester}
\newcommand{\Kurs}{TINF19B3}
\newcommand{\Betreuer}{Prof. Dr.-Ing. Kai Becher}

\newcommand{\DHBWLogoDeckblatt}{\includegraphics[width=4.5cm]{Logos/dhbw-logo}}

\newcommand{\Titel}{Konzeptionierung eines Simulators für 8-bit Prozessoren}
\newcommand{\ArtArbeit}{Studienarbeit}
\newcommand{\Abschluss}{Bachelor of Science}
\newcommand{\Studiengang}{Studiengang Informationstechnik}

\newcommand{\Ort}{Karlsruhe}

%\newcommand{\Abgabedatum}{16.02.2021}


\begin{document}
\onehalfspacing
\pagenumbering{Roman}
	\begin{titlepage}
		{\DHBWLogoDeckblatt}\\[2cm]
		\begin{center}
			\vspace*{-2cm}
			{\Huge \Titel}\\[2cm]
			{\Huge \ArtArbeit}\\[2cm]
			{\Large \Abschluss}\\[0.5cm]
			{\large \Studiengang}\\[0.5cm]
			{\large an der}\\[0.5cm]
			{\large Dualen Hochschule Baden-Württemberg Karlsruhe}\\[0.5cm]
			{\large von}\\[0.5cm]
			{\large\bfseries \Autor}\\[1cm]
			{\large Abgabedatum \today}
			\vfill
		\end{center}
		\begin{tabular}{l@{\hspace{1cm}}l}
			Bearbeitungszeitraum & \Bearbeitungszeitraum \\
			Kurs & \Kurs \\
%			Ausbildungsfirma & \Ausbildungsfirma \\
			Betreuer der Ausbildungsfirma & \Betreuer \\
		\end{tabular}
	\end{titlepage}

\newpage

\thispagestyle{empty}
\begin{center}
\Large\bfseries Erklärung
\end{center}
\medskip
\noindent
Wir versichern hiermit, dass wir unsere \ArtArbeit \ mit
dem Thema: 
\begin{center}
	 \Titel \ 
\end{center}
selbstständig verfasst und keine anderen als die angegebenen Quellen und
Hilfsmittel benutzt haben. Wir versichern zudem, dass die eingereichte elektronische Fassung mit der
gedruckten Fassung übereinstimmt.

\vspace{3cm}
\noindent
\underline{\Ort, \today \hspace{9cm}}\\
%\hfill\underline{\hspace{6cm}}\\
Ort, Datum\hfill Unterschrift\hspace{4cm}

\newpage

\thispagestyle{empty}
\tableofcontents

\newpage

%\thispagestyle{empty}
\thispagestyle{plain}
\cleardoublepage
\addcontentsline{toc}{section}{\listfigurename}
\listoffigures

\addcontentsline{toc}{section}{\listtablename}
\listoftables

\addcontentsline{toc}{section}{Listings}
\lstlistoflistings

\newpage

%\thispagestyle{empty}
\thispagestyle{plain}
\cleardoublepage
\section*{Abkürzungsverzeichnis}
\addcontentsline{toc}{section}{Abkürzungsverzeichnis}


\newpage
\pagenumbering{arabic}

%% Kopf und Fusszeilen==================================================== 
\pagestyle{scrheadings} % Seite mit Headern 

% loescht voreingestellte Stile 
\clearpairofpagestyles
%\clearscrheadings 
\clearmainofpairofpagestyles
%\clearscrplain 

% %%% Kopfzeile 
% einseitig: Bei einseitigem Layout, nur folgende Zeilen verwenden !!! 
\ohead[] {\includegraphics[height=0.5cm]{Logos/Firmenlogokopfzeile}}
\ihead[]{\leftmark} % links: Kapitel
 %\chead[]{} % mitte: 

% %%% Fusszeile 
%\cfoot[]{} % mitte: 
\cfoot[\pagemark]{\pagemark} % rechts: Seitenzahl


% Angezeigte Abschnitte im Header 
\automark{section}  % Inhalt von [\rightmark]{\leftmark} 

\section{Einführung}
Test

\subsection{Ziel der Arbeit}

\newpage

\subsection{Theoretische Grundlagen}

\subsubsection{Architektur eines Prozessors}
\subsubsection{Unterschiede 8-bit, 16-bit, 32-bit und 64-bit Prozessoren}

\subsection{Auswahl der Werkzeuge}
\subsubsection{Programmiersprache}
\subsubsection{GUI-Umgebung}
\subsubsection{Programmierumgebung}

\newpage

\section{Projektplanung}
Platzhalter
\subsection{Zeitplan}
\subsection{Auswahl geeigneter Varianten}
\subsubsection{Intel 8080}
\subsubsection{Anderes Beispiel}

\newpage

\section{Umsetzung}
Platzhalter
\subsection{Abstraktion der Architektur}
\subsubsection{ALU}
\subsubsection{Akkumulator}
\subsubsection{Instruction Register}
\subsubsection{DataBus}
\subsubsection{etc.}
\subsection{Ablauf der Simulation}

\newpage

\section{Fazit}
Platzhalter

\newpage


\addcontentsline{toc}{section}{Literatur}
\begin{thebibliography}{9}
\bibitem{Beispiel}
Google: \url{https://www.google.com}
\end{thebibliography}

%\newpage
%\thispagestyle{empty}
%
%\section*{Anhang}
%\addcontentsline{toc}{section}{Anhang}


\end{document}